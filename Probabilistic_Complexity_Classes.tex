\section{Probabilistische Komplexitätsklassen}


\subsection{Wiederholung: Zeitkomplexität}

\paragraph{Definition}
Sei $M$ eine Turingmaschine, die immer anhält.
Die Funktion $f:\mathbb{N}\to\mathbb{N}$ heißt \emph{Zeitkomplexität} von $M$, wobei $f(n)$ die maximale Anzahl an Berechnungsschritten ist.

\paragraph{Definition}
Seien $f,g:\mathbb{N}\to\mathbb{N}$.
Wir schreiben $f(n) = \mathcal{O}(g(n))$ falls gilt:
\begin{equation*}
	\forall_{n \geq n_0}:\ f(n)\ \leq \ c \cdot g(n), \quad c \in \mathbb{R}^+,\ n_0 \in \mathbb{N}
\end{equation*}

\paragraph{Definition}
$TIME(t(n))\ =\ \bigl\{L\ \bigl\vert\ \exists: \mathcal{O}(t(n))-TM,\ die\ L\ entscheidet \bigl\}$ heißt \emph{Zeitkomplexitätsklasse}.

\paragraph{Definition}
$P = \bigcup\limits_{k=0}^{\infty} TIME(n^k)$ ist die Zeitkomplexitätsklasse der Sprachen, die durch eine deterministische Turingmaschine in polynomieller Zeit entschieden werden können.

\paragraph{Definition}
$NTIME(t(n))\ =\ \bigl\{L\ \bigl\vert\ \exists: \mathcal{O}(t(n))-NTM,\ die\ L\ entscheidet \bigl\}$ heißt \emph{nichtdeterministische Zeitkomplexitätsklasse}.

\paragraph{Definition}
$NP = \bigcup\limits_{k=0}^{\infty} NTIME(n^k)$ ist die nichtdeterministische Zeitkomplexitätsklasse der Sprachen, die durch eine nichtdeterministische Turingmaschine in polynomieller Zeit entschieden werden können.


\subsection{RP}


\subsection{BPP}


\subsection{ZPP}