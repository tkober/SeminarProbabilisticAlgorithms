\section{Probabilistische Komplexitätsklassen}


\subsection{Wiederholung: Zeitkomplexität}

\paragraph{Definition}
Sei $M$ eine Turingmaschine, die immer anhält.
Die Funktion $f:\mathbb{N}\to\mathbb{N}$ heißt \emph{Zeitkomplexität} von $M$, wobei $f(n)$ die maximale Anzahl an Berechnungsschritten ist.

\paragraph{Definition}
Seien $f,g:\mathbb{N}\to\mathbb{N}$.
Wir schreiben $f(n) = \mathcal{O}(g(n))$ falls gilt:
\begin{equation*}
	\forall_{n \geq n_0}:\ f(n)\ \leq \ c \cdot g(n), \quad c \in \mathbb{R}^+,\ n_0 \in \mathbb{N}
\end{equation*}

\paragraph{Definition}
$TIME(t(n))\ =\ \bigl\{L\ \bigl\vert\ \exists TM\ M : M\ entscheidet\ L\ in\ \mathcal{O}(t(n)) \bigl\}$ heißt \emph{Zeitkomplexitätsklasse}.

\paragraph{Definition}
$P = \bigcup\limits_{k=0}^{\infty} TIME(n^k)$ ist die Zeitkomplexitätsklasse der Sprachen, die durch eine deterministische Turingmaschine in polynomieller Zeit entschieden werden können.

\paragraph{Definition}
$NTIME(t(n))\ =\ \bigl\{L\ \bigl\vert\ \exists NTM\ M : M\ entscheidet\ L\ in\ \mathcal{O}(t(n)) \bigl\}$ heißt \emph{nichtdeterministische Zeitkomplexitätsklasse}.

\paragraph{Definition}
$NP = \bigcup\limits_{k=0}^{\infty} NTIME(n^k)$ ist die nichtdeterministische Zeitkomplexitätsklasse der Sprachen, die durch eine nichtdeterministische Turingmaschine in polynomieller Zeit entschieden werden können.


\subsection{RP}
\paragraph{Definition}
Sei $M$ eine zufallsabhängige Turingmaschine und $L = L(M)$. $M$ ist vom Typ \emph{Monte Carlo} falls gilt:
\begin{align*}
	P(M\ akzeptiert\ w) & = 0,\quad w \notin L \\
	P(M\ akzeptiert\ w) & \geq \frac{1}{2},\quad w \in L
\end{align*}
Wir schreiben auch $M$ ist eine \emph{Monte-Carlo-Turinmaschine}.

\paragraph{Definition}
$RP = \bigcup\limits_{k=0}^{\infty} \bigl\{\ L\ \bigl\lvert\ \exists Monte-Carlo-Turingmaschine\ M : M\ entscheidet\ L\ in\ \mathcal{O}(n^k) \bigl\}$ ist die Klasse der Sprachen, die durch eine Monte-Carlo-Turingmaschine in polynomieller Zeit entschieden werden kann und wird \emph{Random Polynomial} genannt.

\paragraph{Aufgabe}
Ist die L Sprache der zufallsabhängigen Turingmaschine aus dem Beispiel \ref{PTM_Example1} in der Klasse $RP$?

\paragraph{Lösung}
$M$ arbeitet unabhängig vom Inhalt des Zufallsbandes in $\mathcal{O}(n)$ und akzeptiert homogene Eingaben mit einer Wahrscheinlichkeit $\geq \frac{1}{2}$.
Allerdings gibt es heterogene Eingaben die mit einer Wahrscheinlichkeit $< \frac{1}{2}$ akzeptiert werden.
Beispielsweise gilt $P(M\ akzeptiert\ w=aab) = 2^{-(3+1)} = \frac{1}{16}$. \\
$\Rightarrow \quad M\ ist\ keine\ Turingmaschine\ vom\ Typ\ Monte\ Carlo$ \\
$\Rightarrow \quad L(M) \notin RP$.

\paragraph{Beobachtung}
Sei $M$ eine Monte-Carlo-Turingmaschine und $L = M(L)$, dann gilt:
\begin{alignat}{2}
	w \notin L \quad & \Rightarrow \quad P(M\ akzeptiert\ w) = 0 \\
	w \in L \quad & \Rightarrow \quad P(M\ akzeptiert\ w) \geq \frac{1}{2}
\end{alignat}
Offenbar wird $w$ eventuell verworfen obwohl $w \in L$ gilt (\emph{false negative}).
Allerdings wird $w$ nie akzeptiert obwohl $w \notin L$ gilt (\emph{false positive}).

\emph{False negatives} können wir nie vermeiden, aber die Wahrscheinlichkeit ihres Auftretens durch Wiederholung der Prüfung $w \in L$ beliebig minimieren.

\paragraph{Satz}
Sei $L \in RP$, $c > 0$.
\setcounter{equation}{0}
\begin{alignat}
	\exists Monte-Carlo-Turingmaschine\ M:\ 
	& L = L(M), \\
	& P(false\ positive) = 0, \\
	& P(false\ negative) \leq c
\end{alignat}

\paragraph{Beweisidee}
Da $L \in RP$ folgt (1) und (2) aus der Definition von $RP$.
Da $M$ eine Turingmaschine vom Typ Monte Carlo ist, gilt per Definition $P(M\ akzeptiert\ w) \geq \frac{1}{2}$ falls $w \in L$. \\
$\Rightarrow \quad P(false\ negative) \leq \frac{1}{2}$ \\
Die Wahrscheinlichkeit, dass $i$ Prüfungen alle ein \emph{false negative} ergeben ist als $\leq 2^{-i}$. 
Durch $\lceil log_{2}(\frac{1}{c}) \rceil$ Wiederholungen der Prüfung gilt also $P(false\ negative) \leq c$. \\
Da $L \in RP$ benötigt auch die wiederholte Prüfung einen polynomiellen Zeitaufwand.

\paragraph{Aufgabe}
Sei $M$ eine Turing-Maschine vom Type Monte Carlo. Wie oft muss geprüft werden ob $w \in L$ gilt, damit die Wahrscheinlichkeit eines \emph{false negatives} nicht größer als eins zu eine Milliarde liegt?

\paragraph{Lösung}
$\lceil log_{2}(\frac{1}{10^{-9}}) \rceil = 30$

\subsection{BPP}


\subsection{ZPP}