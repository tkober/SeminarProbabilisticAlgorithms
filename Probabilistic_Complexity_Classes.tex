\section{Probabilistische Komplexitätsklassen}


\subsection{Wiederholung: Zeitkomplexität}

%\paragraph{Definition}
%Sei $M$ eine Turingmaschine, die immer anhält.
%Die Funktion $f:\mathbb{N}\to\mathbb{N}$ heißt \emph{Zeitkomplexität} von $M$, wobei $f(n)$ die maximale Anzahl an Berechnungsschritten ist.

%\paragraph{Definition}
%Seien $f,g:\mathbb{N}\to\mathbb{N}$.
%Wir schreiben $f(n) = \mathcal{O}(g(n))$ falls gilt:
%\begin{equation*}
%	\forall_{n \geq n_0}:\ f(n)\ \leq \ c \cdot g(n), \quad c \in \mathbb{R}^+,\ n_0 \in \mathbb{N}
%\end{equation*}

%\paragraph{Definition}
%$TIME(t(n))\ =\ \bigl\{L\ \bigl\vert\ \exists TM\ M : M\ entscheidet\ L\ in\ \mathcal{O}(t(n)) \bigl\}$ heißt \emph{Zeitkomplexitätsklasse}.

\paragraph{Definition}
$P = \bigcup\limits_{k=0}^{\infty} TIME(n^k)$ ist die Zeitkomplexitätsklasse der Sprachen, die durch eine deterministische Turingmaschine in polynomieller Zeit entschieden werden können.

%\paragraph{Definition}
%$NTIME(t(n))\ =\ \bigl\{L\ \bigl\vert\ \exists NTM\ M : M\ entscheidet\ L\ in\ \mathcal{O}(t(n)) \bigl\}$ heißt \emph{nichtdeterministische Zeitkomplexitätsklasse}.

\paragraph{Definition}
$NP = \bigcup\limits_{k=0}^{\infty} NTIME(n^k)$ ist die nichtdeterministische Zeitkomplexitätsklasse der Sprachen, die durch eine nichtdeterministische Turingmaschine in polynomieller Zeit entschieden werden können.


\subsection{RP}
\paragraph{Definition}
Sei $M$ eine zufallsabhängige Turingmaschine und $L = L(M)$. $M$ ist vom Typ \emph{Monte Carlo} falls gilt:
\begin{align*}
	Pr[M\ akzeptiert\ w] & = 0,\quad w \notin L \\
	Pr[M\ akzeptiert\ w] & \geq \frac{1}{2},\quad w \in L
\end{align*}
Wir schreiben auch $M$ ist eine \emph{Monte-Carlo-Turinmaschine}.

\paragraph{Definition}
$RP = \bigcup\limits_{k=0}^{\infty} \bigl\{\ L\ \bigl\lvert\ \exists Monte-Carlo-Turingmaschine\ M : M\ entscheidet\ L\ in\ \mathcal{O}(n^k) \bigl\}$ ist die Klasse der Sprachen, die durch eine Monte-Carlo-Turingmaschine in polynomieller Zeit entschieden werden können und wird \emph{Random Polynomial} genannt.

\paragraph{Aufgabe}
Ist die Sprache der zufallsabhängigen Turingmaschine aus dem Beispiel \ref{PTM_Example1} in der Klasse $RP$?

\paragraph{Lösung}
$M$ arbeitet unabhängig vom Inhalt des Zufallsbandes in $\mathcal{O}(n)$ und akzeptiert homogene Eingaben mit einer Wahrscheinlichkeit $\geq \frac{1}{2}$.
Allerdings gibt es heterogene Eingaben die mit einer Wahrscheinlichkeit $< \frac{1}{2}$ akzeptiert werden.
Beispielsweise gilt $Pr[M\ akzeptiert\ w=aab] = 2^{-(3+1)} = \frac{1}{16}$. \\
$\Rightarrow \quad M\ ist\ keine\ Turingmaschine\ vom\ Typ\ Monte\ Carlo$ \\
$\Rightarrow \quad L(M) \notin RP$.

\paragraph{Beobachtung}
Sei $M$ eine Monte-Carlo-Turingmaschine und $L = M(L)$, dann gilt:
\begin{alignat}{2}
	w \notin L \quad & \Rightarrow \quad Pr[M\ akzeptiert\ w] = 0 \\
	w \in L \quad & \Rightarrow \quad Pr[M\ akzeptiert\ w] \geq \frac{1}{2}
\end{alignat}
Offenbar wird $w$ eventuell verworfen obwohl $w \in L$ gilt (\emph{false negative}).
Allerdings wird $w$ nie akzeptiert obwohl $w \notin L$ gilt (\emph{false positive}).

\emph{False negatives} können wir nie vermeiden, aber die Wahrscheinlichkeit ihres Auftretens durch Wiederholung der Prüfung $w \in L$ beliebig minimieren.

\paragraph{Satz}
Sei $L \in RP$, $c > 0$.
\setcounter{equation}{0}
\begin{alignat}
	\exists \exists Monte-Carlo-Turingmaschine\ M:\ 
	& L = L(M), \\
	& Pr[false\ positive] = 0, \\
	& Pr[false\ negative] \leq c, \\
	& M \in \mathcal{O}(n^k) \quad k \in \mathbb{N}^+ 
\end{alignat}

\paragraph{Beweis}
Da $L \in RP$ folgt (1) und (2) aus der Definition von $RP$.
(3) da $M$ eine Turingmaschine vom Typ Monte Carlo ist, gilt per Definition $Pr[M\ akzeptiert\ w] \geq \frac{1}{2}$ falls $w \in L$. \\
$\Rightarrow \quad Pr[false\ negative] \leq \frac{1}{2}$ \\
Die Wahrscheinlichkeit, dass $i$ Prüfungen alle ein \emph{false negative} ergeben ist also $\leq 2^{-i}$. 
Durch $\lceil log_{2}(\frac{1}{c}) \rceil$ Wiederholungen der Prüfung gilt also $Pr[false\ negative] \leq c$. \\
(4) da $L \in RP$ benötigt auch die wiederholte Prüfung einen polynomiellen Zeitaufwand.

\paragraph{Aufgabe}
Sei $L \in RP$. Wie oft muss geprüft werden ob $w \in L$ gilt, damit die Wahrscheinlichkeit eines \emph{false negatives} nicht größer als eins zu eine Milliarde ist?

\paragraph{Lösung}
$\lceil log_{2}(\frac{1}{10^{-9}}) \rceil = 30$

\subsection{BPP}
\paragraph{Definition}
Sei $0 \leq \epsilon < \frac{1}{2}$. Wir sagen \emph{M entscheidet ein wort w mit einem Fehler $\epsilon$} falls gilt:
\begin{align*}
	w \in L \quad & \Rightarrow \quad Pr[M\ akzeptiert\ w] \geq 1 - \epsilon \\
	w \notin L \quad & \Rightarrow \quad Pr[M\ verwirft\ w] \geq 1 - \epsilon
\end{align*}

\paragraph{Definition}
$BPP = \bigcup\limits_{k=0}^{\infty} \bigl\{\ L\ \bigl\lvert\ \exists PTM\ M : M\ entscheidet\ L\ in\ \mathcal{O}(n^k),\ \epsilon =  \frac{1}{3} \bigl\}$ ist die Klasse der Sprachen, die von einer zufallsabhängigen Turingmaschine mit einem Fehler von $\epsilon = \frac{1}{3}$ in polynomieller Zeit entschieden werden können und wird \emph{Bounded-Error Probabilstic Polynomial} genannt.


\subsection{ZPP}
\paragraph{Definition}
Sei $M$ eine zufallsabhängige Turingmaschine und $L = L(M)$. $M$ ist vom Typ \emph{Las Vegas}, wenn sie immer die korrekte Antwort gibt.

\paragraph{Definition}
$ZPP = \bigcup\limits_{k=0}^{\infty} \bigl\{\ L\ \bigl\lvert\ \exists Las-Vegas-Turingmaschine\ M : E(M\ entscheidet\ L) \in \mathcal{O}(n^k) \bigl\}$ ist die Klasse der Sprachen, die durch eine Las-Vegas-Turingmaschine entschieden werden können und für die Zeit bis zum Anhalten einen polynomiellen Erwartungswert hat. 
Sie wird \emph{Zero-Error Probabilitic Polynomial} genannt.


\subsection{Beziehung zwischen ZPP und RP}
\paragraph{Beobachtung}
Offensichtlich gilt $L \in ZPP \Rightarrow \overline{L} \in ZPP$.
Wird nämlich $L$ von einer Las-Vegas-Turingmaschine $M$ mit polynomialem Erwartungswert für den Zeitaufwand akzeptiert, dann wird $L$ von einer modifizierten Las-Vegas-Turingmaschine $M'$ akzeptiert, bei der die Akzeptanz in $M$ durch Anhalten ohne zu akzeptieren, und Anhalten ohne Akzeptanz in $M$ durch Akzeptanz ersetzt wird.

Das $RP$ unter Komplementbildung abgeschlossen ist, ist allerdings nicht offensichtlich.

\paragraph{Definition}
$CoRP = \bigcup \bigl\{\ L\ \bigl\lvert\ \overline{L} \in RP\ \}$

\paragraph{Satz}
$ZPP = RP \cap CoRP$

\paragraph{Beweis}
Zuerst zeigen wir $RP \cap CoRP \subseteq ZPP$.\\
\\
Angenommen $L \in RP \cap CoRP$, dann gibt es für $L$ und $\overline{L}$ Monte-Carlo-Turingmaschinen $M_1$ und $M_2$ in $\mathcal{O}(n^k)$.
Wähle $p(n)$ so, dass $p(n)$ $M_1$ und $M_2$ begrenzt.
Wir entwerfen für $L$ eine Las-Vegas-Turingmaschine $M$ wie folgt:
\begin{enumerate}
	\item $M_1$ akzeptiert $\Rightarrow$ $M$ akzeptiert und hält an.
	\item $M_2$ akzeptiert $\Rightarrow$ $M$ hält an ohne zu akzeptieren.
	\item Wiederhole (1).
\end{enumerate}
Der Erwartungswert für die Ausführungszeit jeder \emph{Runde}(= Schritt (1) und (2)) beträgt $2p(n)$.
Die Wahrscheinlichkeit zu einem Ergebnis zu kommen ist pro Runde $\geq \frac{1}{2}$.
Damit gilt:
\begin{equation*}
	E(M) = 2p(n) + \frac{1}{2}2p(n) + \frac{1}{4}2p(n) + ... = \sum\limits_{k=0}^{\infty}\frac{1}{2^k}2p(n) = 4p(n)
\end{equation*}
$\Rightarrow RP \cap CoRP \subseteq ZPP \hfill \Box$\\
\\
Sehen wir uns nun die Umkehrung an.
Sei $L \in ZPP$.
Wir zeigen $L \in RP$ und $L \in CoRP$.

\paragraph{Teil 1:}
$L \in RP$\\
Wir wissen, es gibt eine Las-Vegas-Turingmaschine $M_A$ mit $L(M_A) = L$ und $E(M_A) = p(n)$.
Wir konstruieren eine Monte-Carlo-Turingmaschine $M_B$ für $L$ wie folgt:
\begin{itemize}
	\item $M_B$ simuliert $w$ auf $M_A$ für $2p(n)$ Schritte, $|w| = n$
	\item $M_A$ akzeptiert $w$ $\Rightarrow$ $M_B$ akzeptiert
	\item sonst verwirft $M_B$
\end{itemize}
Aus dieser Konstruktion folgt:
\begin{align*}
	w \notin L\ &=\ M_A\ akzeptiet\ nicht\ \Rightarrow M_B\ akzeptiert\ nicht \\
	w \in L\ &=\ M_A\ akzeptiet\ ,\ aber\ eventuell\ nicht\ in\ 2p(n)\ Schritten
\end{align*}
Allerdings behaupten wir:
$Pr[M_A\ akzeptiert\ w\ in\ 2p(n)\ Schritten] \geq \frac{1}{2}$ \\
Annahme:
$c = Pr[M_A\ akzeptiert\ w\ in\ 2p(n)\ Schritten] < \frac{1}{2}$ \\
$\Rightarrow\ E(M_A) \geq (1-c)2p(n)$\\
$\xRightarrow{c < \frac{1}{2}}\ 2(1-c) > 1$ und $E(M_A) > p(n) \quad \lightning$\\
$\Rightarrow Pr[M_B\ akzeptiert\ w] \geq \frac{1}{2}$\\
$\Rightarrow M_B\ ist\ vom\ Typ\ Monte\ Carlo$\\
$\Rightarrow L \in RP$

\paragraph{Teil 2:}
$L \in CoRP$\\
Analog mit $\overline{L}$ und $\overline{M_A}$, $\overline{M_B}$.



